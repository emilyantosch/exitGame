\documentclass[usenames,dvipsnames,10pt]{beamer}
\usepackage[utf8]{inputenc}
\usepackage[T1]{fontenc}
\usepackage[german]{babel}
\usepackage{tikz}
\title{Bachelorprojekt - Vorstellung}
\date[ISPN ’80]{Bachelorprojekt - \today}
\author{Emily Antosch \and Karl Döring \and Jonah Mai \and Florian Tietjen}

\usetheme{hawsx}
\setbeamertemplate{section in toc}{%

      \begin{tikzpicture}

        \fill[color=UmUBlue] (1,1) circle (0.2cm);
        \node[anchor=center] (xlab) at (1,1) {\color{white}\inserttocsectionnumber};
        \node[anchor=west] (ylab) at (1.2,0.97) {\color{UmUBlue}\sffamily\textsf{\inserttocsection}};
      \end{tikzpicture}
  }


\begin{document}

    \begin{frame}
        \titlepage
    \end{frame}
    \begin{frame}{Inhaltsverzeichnis}
        \tableofcontents
    \end{frame}
    \section{Organisation}
    \begin{frame}{Organisation}
        \begin{itemize}
            \item Projektleiterin: Emily Antosch
            \pause
            \item Escape-Game für das Labor in Programmieren 2
            \pause
            \item Umsetzung als WebApp
            \pause
            \item Lehrende und Lernende unterstützen
        \end{itemize}
    \end{frame}
    \section{Umsetzung}
    \begin{frame}{Umsetzung}
      \begin{itemize}
        \item Um unser Ziel zu erreichen, haben wir einige Tools ausgewählt
        \item Erfahrung, Beliebtheit und Eignung für die Aufgaben
        \item Verbesserung über Vanilla JS/HTML/CSS
        \item Lernerfolg der Teilnehmer für das weitere Leben
      \end{itemize}
    \end{frame}
    \begin{frame}{React.js}
        \begin{itemize}
            \item Hauptframework (Frontend)
            \item Übernimmt Hauptaufgaben wie DOM, Rendering und Hooks
            \item Beliebtestes Frontend-Framework, deshalb auch viel Support
            \item Erlaubt viel Integration mit anderen Tools
            \item Läuft mit TypeScript/JSX
        \end{itemize}
    \end{frame}
    \begin{frame}{TailwindCSS}
        \begin{itemize}
            \item CSS-Framework, was die Designarbeit vereinfacht
            \item Verlassen von JSX/HTML nicht nötig
            \item Erweiterbar und Anpassbar auf unser Design
            \item Modern, leicht
        \end{itemize}
    \end{frame}
    \begin{frame}{Firebase/UserAuth}
            \begin{itemize}
                \item Um die User für die einzelnen Labore freizuschalten und in eine Datenbank zu schreiben, möchten wir Firebase verwenden
                \item Leichter als eine eigene Datenbank, wie MongoDB oder MySQL
                \item User Authentication ziemlich leicht
                \item Ist nicht ganz ohne Kosten, allerdings bleibt dies für wenige Anfragen extrem wahrscheinlich
            \end{itemize}
    \end{frame}
    \begin{frame}{Git/Github}
            \begin{itemize}
                \item Um die Pipeline möglichst einfach und fehlerunanfällig zu machen, sollte Version Control verwendet werden
                \item Git ist mit Abstand das einfachste, mächtigste und vielseitige Programm für diesen Zweck
                \item Github ist der größte Anbieter für Online-Repositories für Teams in der Software-Entwicklung
                \item Großer Lernfaktor, Git ist essenzieller Teil von fast jeder Pipeline
                \item Mögliches Hosting über Github Pages
            \end{itemize}
    \end{frame}
\end{document}
