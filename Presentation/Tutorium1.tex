\documentclass[usenames,dvipsnames,10pt]{beamer}
\usepackage[utf8]{inputenc}
\usepackage[T1]{fontenc}
\usepackage[german]{babel}
\usepackage{tikz}
\title{Tutorium 1 - Programmieren 1}
\date[ISPN ’80]{PR/01 - \today}
\author{Eric Antosch}

\usetheme{hawsx}
\setbeamertemplate{section in toc}{%
  
      \begin{tikzpicture}
        
        \fill[color=UmUBlue] (1,1) circle (0.2cm);
        \node[anchor=center] (xlab) at (1,1) {\color{white}\inserttocsectionnumber};
        \node[anchor=west] (ylab) at (1.2,0.97) {\color{UmUBlue}\sffamily\textsf{\inserttocsection}};
      \end{tikzpicture}
  }


\begin{document}
    
    \begin{frame}
        \titlepage
    \end{frame}    
    \begin{frame}{Inhaltsverzeichnis}
        \tableofcontents
    \end{frame}
    \section{Begrüßung}
    \begin{frame}{Begrüßung}
        \begin{itemize}
            \item Emily Antosch
            \pause
            \item 21 Jahre alt
            \pause 
            \item Elektro- und Informationstechnik, dual
            \pause
            \item 4. Semester
        \end{itemize}
    \end{frame}
    \section{Organisation}
    \begin{frame}{Organisation}
        \begin{itemize}
            \item Termine: Dienstags, 17:00 - 19:00
            \item Zusammenfassung der Vorlesung
            \item Fragerunde zu den Themen des Moduls
            \item Extraaufgaben mit Besprechung
            \item Unterlagen finden Sie im Emil-Raum
        \end{itemize}
    \end{frame}
    \section{Zusammenfassung}
    \begin{frame}{Zusammenfassung}
        \begin{block}{Sie haben schon gelernt...}
            \begin{itemize}
                \item was ein Von-Neumann-Rechner ist,
                \item wofür man eine IDE verwendet,
                \item wie man höhere Programmiersprachen und deren Übersetzung charakterisiert
                \item und wie man mit Visual Studio umgeht
            \end{itemize}
        \end{block}
    \end{frame}
    \begin{frame}{Zusammenfassung}
        \begin{block}{Sie haben schon gelernt...}
            \begin{itemize}
                \item wie ein Programm in C aufgebaut ist,
                \item was Kommentare sind und wie man sie schreibt,
                \item wo der Unterschied zwischen Quelldateien und Headerdateien ist
                \item und wie man mit der Funktion printf() umgeht.
            \end{itemize}
        \end{block}
    \end{frame}
    \begin{frame}{Zusammenfassung}
        \begin{block}{Sie haben schon gelernt...}
            \begin{itemize}
                \item was Variablen sind und welche Datentypen es gibt,
                \item wie man Variablen und Konstanten deklariert und initialisiert,
                \item was der Scope einer Variablen ist und wie man diesen bestimmt
                \item und was die ASCII-Tablle ist und wie man sie verwendet.
            \end{itemize}
        \end{block}
    \end{frame}
    \begin{frame}{Zusammenfassung}
        \begin{block}{Sie haben schon gelernt...}
            \begin{itemize}
                \item was Bedingungen sind und wo der Unterschied zwischen if und switch ist,
                \item was Schleifen sind und wo der Unterschied zwischen for, while und do...while ist,
                \item was Sprunganweisungen sind und wie man goto und label verwendet,
                \item und wie man die Kontrollstrukturen sinnvoll miteinander verschachtelt.
            \end{itemize}
        \end{block}
    \end{frame}
    \begin{frame}{Zusammenfassung}
        \begin{block}{Sie haben schon gelernt...}
            \begin{itemize}
                \item was Funktionen sind und wie man sie definiert,
                \item was Headerdateien sind, welche es gibt und wie man eigene erstellt,
                \item wo der Unterschied zwischen Call-By-Value und Call-By-Reference ist
                \item und was Rekursion ist und wie man diese erstellt.
            \end{itemize}
        \end{block}
    \end{frame}
    \section{Fragerunde}
\framepic{Idee.jpg}{
	%\framefill
    \textcolor{white}{Fragen?}
    \vskip 0.5cm
}

    \section{Aufgaben}
    \begin{frame}{Aufgaben}
        \begin{block}{printf()}
            Nennen sie die gängigsten Format Identifier, die man zusammen 
            mit printf() nutzt.
        \end{block}
    \end{frame}
    \begin{frame}{Aufgaben}
        \begin{block}{IDE}
            Wofür steht die Abkürzung IDE und was beinhaltet ein solches Programm? Welche 
            IDEs kennen Sie?
        \end{block}
    \end{frame}
    \begin{frame}{Aufgaben}
        \begin{block}{Kommentare}
            Welche zwei Arten von Kommentaren kennen Sie und wo liegt der Unterschied?
        \end{block}
    \end{frame}
    \begin{frame}{Aufgaben}
        \begin{block}{Scope}
            Was bezeichnen wir als Scope, auf welche Elemente in unserem Programm bezieht er sich und
            wie gehen wir damit um?
        \end{block}
    \end{frame}

    \begin{frame}{Aufgaben}
        \begin{block}{Funktionen}
            Wie definieren wir eine Funktion? Wie nennen wir die einzelnen Bausteine, die wir dafür brauchen?
        \end{block}
    \end{frame}
    \begin{frame}{Aufgaben}
        \begin{block}{Datentypen}
            Welche Datentypen haben wir in C? Geben Sie die Datentypen geordnet nach der Speicherkapazität von
            groß nach klein an.
        \end{block}
    \end{frame}

    

\end{document}
